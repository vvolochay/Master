\startconclusionpage

Результатом данной работы являются достаточно точные оценки на работу несмещенных операторов для несмещенной black-box сложности. Были установлены верхняя и нижняя оценка на работу тернарного оператора. Так же было установлено, что необходимости рассматривать операторы больше нет, то есть при k = 3 достигается оптимальное значение. Тернарный оператор можно применять к некоторым классам задач, так как алгоритм, использующий таковые, работает не хуже оптимальных детерминированных алгоритмов.  

Так же были проделаны исследования бинарного и унарного операторов. Были установлены некоторые ограничения на работу бинарного оператора, а так же предложен алгоритм работы с бинарным оператором. Более точные оценки еще предстоит получить.

Был разработан алгоритм работы унарного оператора, а так же получение всего пространства поиска для алгоритма, использующего таковой. Результатом исследования унарного оператора стала верхняя оценка на количество запросов данного алгоритма к пространству поиска. 

Полученные оценки явно показывают, что задачу Needle можно разрешить быстрее, чем простым перебором. Идеи, использованные в разработанных алгоритмах, могут быть использованы в будущем для других задач, к примеру, разрешения плато, которое встречается во многих оптимизационных задачах.

Среди открытых вопросов остаются нижние оценки для бинарных и унарных операторов.