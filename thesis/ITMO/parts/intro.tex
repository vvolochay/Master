\startprefacepage

Одним из основных предназначений эволюционных алгоритмов является решение оптимизационных задач. Однако же эволюционные алгоритмы относятся к методам вероятностого поиска, то есть найти точное время работы алгоритма не представляется возможным. Они не всегда могут найти решение за отведенное время, однако при гораздо меньших временных затратах могут выдать достаточно хорошее решение. 

Существует множество работ, анализирующих время работы для тех или иных задач и алгоритмов для их решения. Однако, для методов вероятностого поиска плохо подходят классические подходы оценки сложности. 


Допустим, дана некоторая задача оптимизации, определенная как некоторая функция, определенная на некотором пространстве поиска. Известно, что существует алгоритм, решающий данную задачу. При этом имеется неизвестный экземпляр этой задачи. Алгоритм не имеет знаний о решаемой задаче, и может получать информацию об экземпляре только путем конструирования точки из пространства поиска с помощью вычисления функции приспособленности, по возможности ориентируясь на уже полученные результаты запросов. 
В таком случае хотелось бы оценить, как быстро можно эту задачу решить в принципе. При этом время решения будет равно количеству запросов данной функции приспособленности.


Именно по такому принципу работают эволюционные алгоритмы. При этом понимание, в каких случаях работают эволюционные алгоритмы, а в каких - нет и анализ времени работы позволяет разрабатывать новые, более эффективные алгоритмы. В таком случае появляется необходимость появления модели теории сложности для методов вероятностного поиска. 


Модель несмещенного черного ящика (далее black-box) впервые введена в 2010 году \cite{1} и в настоящее время является одной из станартных моделей сложности в эволюционных вычислениях. В частности, унарная несмещенная модель дает более реалистичную оценку сложности для ряда функций, чем исходная неограниченная черным ящиком модель в статье \cite{2}.   
Важным преимуществом несмещенной модели является то, что она позволяет анализировать влияние арности использованных для выборки операторов. Кроме того, новые точки поиска могут быть выбраны либо только равномерно случайным образом, либо из распределений, которые зависят от ранее созданных несмещенно точек поиска. 


  В Главе 1 представлено описание предметной области данной работы: задача Needle, теоретический анализ работы эволюционных алгоритмов, модель несмещенной black-box сложности и несмещенные операторы. 
  
  В Главе 2 приведено подробное описание применения к исходной задачи несмещенной black-box модели.
  
  В Главе 3 рассматривается применение к задаче Needle несмещенных операторов разной арности, описаны сами операторы и оценки времени сложности работы алгоритмов, использующие их. 
  
  
  
  