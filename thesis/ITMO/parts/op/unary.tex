\label{unary}

%описание оператора и принципа
В силу того, что унарный несмещенный оператор может изменять только один случайный бит в векторе, в модели генетических алгоритмов он представляет собой оператор мутации. 

(Не хватает немного)

Разобъем все пространство поиска на некоторые классы векторов, где каждый шаг - это расстояние между первоначальным вектором и векторами этого множетсва, то есть, количество отличающихся бит. Классы будут генерироваться слеующим образом: 
\begin{itemize}
   \item $q_0$ $\leftarrow$ random
   \item $D_1$ - класс решений, отличающихся от первоначальной особи на 1 бит, размер множества n
   \item $D_2$ - размер ${n}\choose{2}$
   \item $D_3$ ...
   \item ...
   \item $D_{n-1}$ - размер ${n}\choose{n - 1}$ = n
   \item $D_n$ - размер 1, \bar{q_0}
\end{itemize} 


\newtheorem{theorem}{Теорема}
\begin{myth}
Даны непересекающиеся множества $S_1, ... S_k$ размеров $n_1 ... n_k$.
В одном из множеств находится элемент x. Утверждается, что в этой постановке при отсутствии другой информации, оптимизационный алгоритм выглядит так: 
\begin{algorithm}[H]
\caption{Унарный алгоритм}\label{lst1}
\begin{algorithmic}
        \For{i = 0, ... , k}
	    \State Храним $t_1 ... t_k$ - сколько запросов было сделано к соответствующему множеству
	    \State На определенном шаге вычисляем $E[z_i] = (\frac{n_i - 1}{n_i})^{t_i}$
	    \State Выбираем множество i c максимальным $E[z_i]$ и делаем запрос к этому множеству.
	    \EndFor
\end{algorithmic}
\end{algorithm}
\end{myth}
\begin{proof}
    Пусть мы знаем, сколько уникальных элементов $u_1...u_k$ было запрошено из каждого множества. Известно что количество элементов будет меньше или равно числу запросов к множеству. ($u_i \leq t_i$). 
    
    Пусть x - искомый экземпляр данной задачи, находится соответственно в одном из множеств. Предположим, что элемент x раньше не возвращался из запросов (так как при возвращении искомого элемента функция бы завершилась).
    
    
    $P_i$ = [вероятность, что $x$ лежит в $i$] = $\frac{t_i - u_i}{\sum_{j}{t_j - u_j}}$ 
    
    $p_i$ = [вероятность, что на запросе к i вернут k] = $\frac{1}{n_i} \cdot P_i$ = $\frac{n_i - u_i}{\sum_{j}{t_j - u_j}}$
    
    Так как $u_i$ - величина переменная а каждом шаге, можем посчитать матожидание в таком формате: $u_i \sim E(u_i(n_i, t_i))$
    
    До запроса было x уникальных запросов, после запроса будет: $$x + \frac{n - x}{n} = 1 + x \cdot (1 - \frac{1}{n_i})$$ 
    
    Таким образом: $E[u(n_i, t_i)] = 1 + E[u_i(n_i, t_{i} - 1)] \cdot (1 - \frac{1}{n_i})$
    
    Для $ t_i = 0 \to 0 $ - база индукции.
    
    Рассмотрим переход: 
    $E[u(n_i, t_i)]$ = $\frac{{n_i}^{ti} - {(n_i - 1)}^{ti}}{{n_i}^{t_i - 1}}$  = 1 + $[ \frac{{n_i}^{ti} - {(n_i - 1)}^{t_i}}{{n_i}^{t_i}} ] \cdot (\frac{1}{n_i})$ = $ \frac{{n_i}^{ti + 1} - {(n_i - 1)}^{t_i + 1}}{{n_i}^{t_i}} $ 
    
    Таким образом мы получаем значение, равное количеству строк на следующем шагу, а значит, теорема доказана.

\end{proof}

%здесь будет вставка от Дена для увеличения объема работы. Не знаю, пригодится ли дальше это, но пока пусть будет.

(доделать доказательство, доделать разбор вероятностей по классам)
