\chapter{Обзор предметной области}
\label{chapter1}

В этой главе приводится описание работы, обзор предметной области, основные термины, встречающиеся в тексте. 

\section{Теория сложности эволюционных вычислений}
Рандомизированный поиск эвристик является одним из основных подходов для решения задач оптимизации. Практическим путем было установлено, что рандомизированный подход часто является довольно успешным, и таким образом широко используется. Его основное преимущество заключается в том, что нет необходимости в глубоком понимании самой оптимизационной задачи, и полученные походы после легко могут быть повторно переиспользованы на аналогичных задачах.

Одна из проблем таких эвристик заключается в том, что трудно предсказать, какая из задач легкоразрешима поиском рандомизированных эвристик, а какая трудно. Тем не менее, для понимания того, какие задачи легкие, а какие трудные, была весьма необходима теория сложности похожая на то, что существует в классической алгоритмике. Таким образом, модель black-box сложности была первой попыткой ввести подобную теорию сложности была представлена в работе \cite{2}. 


Было установлено, что для многих задач наилучший black-box алгоритм работает быстрее, чем существующие эволюционные подходы. К примеру, алгоритм (1 + 1), который оптимизирует гораздо медленнее. Таким образом, одна из целей работы — рассмотреть классы алгоритмов с несмещенными [6] операторами различных арностей: unary(мутации), binary(кроссовер) и т. д.

\subsection{Несмещенная black-box сложность}
В этой части определена одна из моделей и black-box сложности, а именно несмещенная модель.

%много какой-то неуместной воды. Отфильтровать.

%Формально определим black-box сложность. Мы будем
%оценивать исключительно число запросов к оракулу, считая все %остальные вычисления бесплатными. Время работы TA,f вероятностного %алгоритма A для функции f — это ожидаемое число запросов до момента %нахождения точки оптимума.

Особый интерес в рамках данной работы представляет собой несмещенная модель black-box сложности. Модель определяет класс aлгoритмoв. Таким образом, сложность класса определяется относительно алгоритмов в представленной модели.

Пусть существует некоторый класс функций F, который вполне известен алгоритму. Возьмем некоторую функцию f, такую что $f \in F$ и скрываем от алгоритма. Алгоритм может получать информацию о функции только с помощью запросов к черному ящику о значениях f в некоторых точках пространства. Задача алгоритма заключается в том, чтобы найти некий глобальный оптимум функции.


\begin{definition}
(Несмещенный оператор арности k [1]). Возьмем некоторый
$k \in  N$ и зафиксируем. Несмещенным распределением арности k. $ D (\cdot | x^{1}, ..., x^k)$  называется такое вероятностое распределение над $\{0, 1\}^{n}$, что для любых $y,z \in \{0, 1 \}^n$ и перестановки $\sigma$ и размера $n$ выполняются следующие два условия:
 $$ 1. D ( y | x^{1},..., x^{k} ) = D(\sigma(y) | \sigma(x^{1}),...., \sigma(x^{k})).$$

 $$ 2. D ( y | x^{1},..., x^{k} ) = D(y \oplus z | x^{1} \oplus z,...., x^{k} \oplus z);$$ 

\end{definition}

Первое свойство в определении является — инвариантностью относительно перестановки, второе свойство - инвариантоность относительно побитового ИЛИ. Несмещенная модель арности k содержит алгоритмы, работающие по принципу следующего алгоритма:

\begin{algorithm}[H]
\caption{Black-box алгоритм в несмещенной модели}\label{lst1}
\begin{algorithmic}
        \State Выбрать $x^0$ равномерно случайно из $\{0,1\}^n$ и запросить $f(x^0)$  
		\For{t = 1,2,3... \textrm{до окончания работы} }
	    \State В зависимости от $f(x^0),...,f(x^{t-1})$ выбрать несмещенный оператор порядка $k p^t$ и $k$ запрошенных ранее точек $x^i,...,x^{ik}$
	    \State Выбрать $x^t$ из $p^{t} (x^{i_1},...,x^{i_k})$ и запросить $f(x^t)$
		\EndFor
\end{algorithmic}
\end{algorithm}

Определим black-box сложность. В певую очередь будем оценивать число запросов к черному ящику, пренебрегая сложностью всех остальных запросов. Время работы T_{A,f} вероятностного алгоритма A для функции f — это ожидаемое число запросов до момента нахождения глобального оптимума.

\begin{definition}
(Black-box сложность). 
Сложность класса функций $\mathcal{F}$ относительно класса алгоритмов $A$ определяется как
$$T_{A}(\mathcal{F}) = \min_{A \in \mathcal{A}} \max_{f \in \mathcal{F}} T_{A, f} $$
\end{definition}



\chapterconclusion
В данной главе были представлены основные определения и понятия, необходимые для постановки задачи и дальнейшего рассмотрения её и подходов к её решению. Основным выводом главы считается постановка определения несмещенной black-box модели, нахождение верхних и нижних оценок которой является основной целью работы.   
