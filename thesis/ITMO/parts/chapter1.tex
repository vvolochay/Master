\chapter{Обзор предметной области}
\label{chapter1}

В этой главе приводится описание работы, обзор предметной области, основные термины, встречающиеся в тексте. Основная задача этой главы - изложение теории вычислительной сложности из которой вытекает описание алгоритма черного ящика, а так же black-box сложности с её моделями, особенно акцентируя внимания на несмещенной модели.

\section{Теория сложности эволюционных вычислений}
Рандомизированный поиск эвристик является одним из основных подходов для решения задач оптимизации. Практическим путем было установлено, что рандомизированный подход
часто является довольно успешным, и таким образом широко используется. Его основное преимущество заключается в том, что нет необходимости в глубоком понимании самой
оптимизационной задачи, и полученные походы после легко могут быть повторно переиспользованы на аналогичных задачах.

Одна из проблем таких эвристик заключается в том, что трудно предсказать, какая из задач легкоразрешима поиском рандомизированных эвристик, а какая трудно. Тем не менее, для понимания того, 
какие задачи легкие, а какие трудные, была весьма необходима теория сложности похожая на то, что существует в классической алгоритмике. Таким образом, модель black-box сложности была первой 
попыткой ввести подобную теорию сложности была представлена в работе~\cite{2}. 

Было установлено, что для многих задач наилучший black-box алгоритм работает быстрее, чем существующие эволюционные подходы (к примеру, алгоритм $(1+1)$, который оптимизирует гораздо медленнее). 
Таким образом, одна из целей работы~--- рассмотреть классы алгоритмов с несмещенными~\cite{6} операторами различных арностей: unary (мутации), binary (кроссовер) и~т.д.

\subsection{Black-box сложность}

В этой части определена одна из моделей и black-box сложности, а именно несмещенная модель.
Black-box сложность изучает, сколько оценок функции необходимо при ожидании, пока оптимальный алгоритм черного ящика первый раз вернет оптимальное решение задачи.
Как правило, рандомизированные алгоритмы поиска эвристики, как и эволюционные алгоритмы, являются подобными оптимизаторами с черным ящиком: они чаще всего не зависят
от задачи и, как таковые, узнают о задаче только когда она решена путем создания и оценки точек множества поиска.

Оценка black-box сложности задачи обычно является оценкой фитнес-функции, необходимой для любой поисковой эвристики для решения задачи.

Существует несколько моделей black-box сложности: неограниченная, несмещенная, компараторная b т.д. Неограниченная модель никак не ограничевает никаким образом процедуру отбора и выбора алгоритмов, можно использовать любой алгоритм, работающий с экземпляром как с черным ящиком. В компараторной модели значения функции можно только сравнивать, а сами значения использовать нельзя. А так же существует несмещенная модель, в которой можно использовать алгоритм, манипулирующий с запросами с помощью только несмещенных операторов, которые будут формально определены в работе позднее. В основном, множество эволюционных алгоритмов несмещенные (за исключением некоторых операторов кроссовера). Так же алгоритм не должен различать значения $0/1$ и индексы битов.

\subsection{Несмещенная black-box сложность}
Как уже упоминалось, некоторые недостатки неограниченной black-box модели вдохновили Лехре и Витт~\cite{1} на то, чтобы ввести новую, модель черного ящика c более ограничительным характером, как и 
появилась несмещенная black-box модель. Она остнована на наблюдении, что многие рандомизированные алгоритмы поиска эвристики используют только несмещенные операторы вариаций. 
В этой части будет кратко представлена несмещенная black-box модель. 

Несмещенная модель black-box сложности  представляет собой особый интерес в рамках данной работы. Модель определяет класс aлгoритмoв. Таким образом, сложность класса определяется относительно алгоритмов
в представленной модели.

Пусть существует некоторый класс функций $F$, который известен алгоритму. Возьмем некоторую функцию $f$, такую что $f \in F$, и скрываем от алгоритма. Алгоритм может получать информацию о функции 
только с помощью запросов к черному ящику о значениях $f$ в некоторых точках пространства поиска. Задача алгоритма заключается в том, чтобы найти некий глобальный оптимум функции.

%Формально определим black-box сложность. Мы будем
%оценивать исключительно число запросов к оракулу, считая все %остальные вычисления бесплатными. Время работы TA,f вероятностного %алгоритма A для функции f — это ожидаемое число запросов до момента %нахождения точки оптимума.


\begin{definition}
(Несмещенный оператор арности $k$~\cite{1}). Возьмем некоторое $k \in  N$ и зафиксируем. Несмещенным распределением арности $k$ $D(\cdot | x^{1}, ..., x^k)$  называется такое вероятностое распределение 
над $\{0, 1\}^{n}$, что для любых $y,z \in \{0, 1 \}^n$ и перестановки $\sigma$ и размера $n$ выполняются следующие два условия:
\begin{enumerate}
\item $D ( y | x^{1},..., x^{k} ) = D(\sigma(y) | \sigma(x^{1}),...., \sigma(x^{k}))$;
\item $D ( y | x^{1},..., x^{k} ) = D(y \oplus z | x^{1} \oplus z,...., x^{k} \oplus z)$. 
\end{enumerate}
\end{definition}

Первое свойство в определении является инвариантностью относительно перестановки, второе свойство~--- инвариантоность относительно побитового ИЛИ. Несмещенная модель арности $k$ содержит алгоритмы, 
работающие по принципу следующего алгоритма:


Грубо говоря, непредвзятый оператор должен относиться к битовым позициям $1, ..., n$ и к разрядным записям 0 и 1 несмещенным способом. В частности, несмещенный оператор не может потребовать определенного значения бита, которое будет установлено равным 0 или 1. Как уже упоминалось, несмещенная модель (в дополнение к некоторым весьма искусственным алгоритмам) допускает понятие арности. Несмещенным k-арным black-box алгоритмом является тот, который использует только такие операторы переворота и  принимающий до k аргументонтов.


\begin{algorithm}[H]
\caption{Black-box алгоритм в несмещенной модели}\label{lst1}
\begin{algorithmic}
        \State Выбрать $x^0$ равномерно случайно из $\{0,1\}^n$ и запросить $f(x^0)$  
		\For{t = 1,2,3\ldots \textrm{до окончания работы} }
	    \State В зависимости от $f(x^0),...,f(x^{t-1})$ выбрать несмещенный оператор порядка $k p^t$ и $k$ запрошенных до этого точек $x^i,\ldots,x^{ik}$
	    \State Выбрать $x^t$ из $p^{t} (x^{i_1},\ldots,x^{i_k})$ и запросить $f(x^t)$
		\EndFor
\end{algorithmic}
\end{algorithm}

Определим black-box сложность. В первую очередь будем оценивать число запросов к черному ящику, пренебрегая сложностью всех остальных запросов. Время работы $T_{A,f}$ вероятностного алгоритма $A$ для 
функции $f$~--- это ожидаемое число запросов до момента нахождения глобального оптимума.

\begin{definition}
(Black-box сложность). 

Сложность класса функций $\mathcal{F}$ относительно класса алгоритмов $A$ определяется как
$$T_{A}(\mathcal{F}) = \min_{A \in \mathcal{A}} \max_{f \in \mathcal{F}} T_{A, f}.$$
\end{definition}

Обратим внимание, что алгоритм из Листинга 1 работает всегда. Это обусловлено тем, 

\chapterconclusion
В данной главе были представлены основные определения и понятия, необходимые для постановки задачи и дальнейшего рассмотрения её и подходов к её решению. Основным выводом главы считается постановка 
определения несмещенной black-box модели, нахождение верхних и нижних оценок которой является основной целью работы.   
