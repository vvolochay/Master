\chapter{Верхние и нижние оценки}
\label{chapter3}

\section{Оценка сложности алгоритмов с несмещенными операторами}

Чтобы исключить алгоритмы, поведение которых сильно отличается от естественных эвристик, используемых в оптимизации, можно ввести
дополнительные ограничения. Несмещенная модель,
вводит два существенных ограничения. Во-первых, решения, принимаемые алгоритмом, могут зависеть только от наблюдаемых значений функции, но не от самих точек. Во-вторых, алгоритм может делать только запросы,
полученные при помощи несмещенных операторов, определенных ниже.

Представляет интерес класс несмещенных алгоритмов арности K. Под
несмещенными алгоритмами (произвольной арности) понимаются алгоритмы
следующего вида:

\begin{itemize}
   \item Запрос $Q_1$ - запрос, сгенерированный случайным образом.
   \item Запрос $Q_{i+1}$ - реализация случайной величины, такой что:
        \begin{itemize}
            
            \item  Для любой z $$ p(Q_{i+1} | Q_1, ... Q_i, A_1, ... A_i) = p(Q_{i+1} \oplus z | Q_1 \oplus z, ..., Q_i \oplus z, A_1, ..., A_i)$$  
            
            \item Для любой перестановки perm() 
            $  p(Q_{i+1} | Q_1, ..., Q_i, A_1, ..., A_i) = p(perm(Q_{i+1}) |perm(Q_1), ..., perm(Q_i), A_1, ... A_i) $ 
            
        \end{itemize}

 \item Алгоритм арности k: случайная величина, генерирующая $Q_{i+1}$ рандомизировано выбирает не более k различных индексов $H_j$ из $[1; i]$ и генерирует $Q_{i+1}$ используя только $Q_{H_1}, ..., Q_{H_K}, A_{H_1}, ... A_{H_K}$
  \end{itemize}
  
  
Иными словами, функция, генерирующая запрос, не меняется, если индексы
всех битовых строк переставить одинаковым образом, или если все
битовые строки поксорить с константой.


 \begin{algorithm}[H]
 \caption{Black-box алгоритм в несмещенной модели}
 \label{lst:ea}
 \begin{algorithmic}[1]
    \STATE{Выбрать $x^0$ равномерно случайно из $\{0,1\}^n$ и запросить $f(x^0)$ } \\
    \For{t = 1,2,3... \textrm{до окончания работы} }{
    \textrm{В зависимости от $f(x^0),...,f(x^{t-1})$ выбрать несмещенный оператор порядка $k p^t$ и $k$ запрошенных ранее точек $x^i,...,x^{ik}$ } \\
    \textrm{Выбрать $x^t$ из $p^{t} (x^{i_1},...,x^{i_k})$ и запросить $f(x^t)$}
    }
    \EndFor
\end{algorithmic}
\end{algorithm}


\subsection{Тернарный алгоритм}
В качетве операторов для данного алгоритма введем следующие несмещенные операторы: 
\begin{itemize}
   \item $flip$ - флип одного бита
   \item $flip'(a, b)$ - флипнуть один бит из совпадающего суффикса
   \item $xor3(a,b,c)$ - оператор исключающего ИЛИ
\end{itemize}
\subsection{Бинарный алгоритм}
\subsection{Унарный алгоритм}
\section{Выводы по главе 3}