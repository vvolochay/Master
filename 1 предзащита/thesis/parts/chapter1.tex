\chapter{Обзор предметной области}
\label{chapter1}

\section{Теория сложности эволюционных вычислений}
Теория сложности ставит своей целью определение сложности
вычислительных задач. В классической теоретической информатике
плодотворное взаимодействие теории сложности, изучающей, какие
минимальные вычислительные затраты необходимы для решения задачи,
и теории алгоритмов, дающей решение и тем самым показывающей, каких
вычислений достаточно, является одной из ключевых сил, развивающих
область.
В случае эволюционных алгоритмов и других методов вероятностного
поиска уже существует немало результатов со стороны теории алгоритмов. 
\subsection{Black-box сложность}

Было установлено, что для многих задач наилучший black-box алгоритм работает быстрее, чем существующие эволюционные подходы. К примеру, алгоритм (1 + 1), который оптимизирует гораздо медленнее. Таким образом, одна из целей работы — рассмотреть классы алгоритмов с несмещенными [6] операторами различных арностей: unary(мутации), binary(кроссовер) и т. д.

Формально определим black-box сложность. Мы будем
оценивать исключительно число запросов к оракулу, считая все остальные
вычисления бесплатными. Время работы TA,f вероятностного алгоритма A
для функции f — это ожидаемое число запросов до момента нахождения
точки оптимума.

\section{Задача Needle}

Задача Needle сформулирована следующим образом: 

Для всех $z \in \{0, 1\}^n \;\;\; $  
    \begin{math} 
    f_{z} : \{0, 1 \}^n \rightarrow \{0,1\}; x \rightarrow  \left\{ \begin{array}{ll}
    1 & \textrm{$x = z$}\\
    0 & \textrm{$x \ne z$}
    \end{array} \right.
    \end{math}

Needle является одной из сложных задач, так как за каждый запрос алгоритм получает лишь знание, совпадает ли запрос с оптимумом или нет, и никаких выводов о других запросах сделать не может. Лучший детерминированный алгоритм, решающий задачу, просто делает все возможные запросы по порядку, и среднее число запросов до оптимума  (2n + 1) / 2. Такое же время работы у алгоритма, генерирующего случайную перестановку. Таким образом, новые лучшие полученные оценки должны быть более точными.

\section{Выводы по главе 1}

