\startprefacepage

Одним из способов применения эволюционных алгоритмов является решение оптимизационных задач. Одной из таких задач является задача Needle. Для получения улучшенных оценок сложности работы алгоритма, рассмотрим задачу в модификации, основанной на идеях black-box сложности.  


% как-то добавить и переформулировать
  Модель непредвзятый черного ящика введена в Lehre и Уитт (2012) является в настоящее время одним из стандартные модели сложности в эволюционных вычислениях. В частности, унарным непредвзятый модель дает более реалистичную оценку сложности для ряда функций, чем исходного неограниченного черного ящика модели Droste и др. (2006). Важным преимуществом несмещенной модели является то, что она позволяет нам анализировать влияние арностью операторов выборки в использовании. Кроме того, новые точки поиска могут быть выбраны либо только равномерно случайным образом или из распределений, которые зависят от ранее созданных точек поиска непредвзято. В этом разделе мы дадим краткое определение модели черного ящика беспристрастной, указывая на Lehre и Уитт (2012) и Doerr и Winzen (2014b) для более подробного введения
  
  