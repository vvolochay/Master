\chapter{Описание предлагаемого подхода}
\label{chapter2}

\section{Рассматриваемая оптимизационная задача}
\subsection{Описание подхода}
Будем рассматривать классы алгоритмов для операторов на несмещенных сложностях. Решения, принимаемые алгоритмом, могут зависеть только от наблюдаемых значений функции, но не от самих точек. Так же алгоритм может делать  запросы, полученные только при помощи несмещенных операторов, обладающих двумя свойствами: инвариантностью относительно перестановки и инвариантностью относительно побитового исключающего или.
	
Наибольший интерес представляют собой алгоритмы для операторов  несмещенных сложностей арности k = 1, 2, 3.

Время работы black-box алгоритма оценивается числом запросов к черному ящику,  то есть это ожидаемое число запросов до точки  нахождения оптимума.
	
Подсчет верхних оценок сводится к разработке алгоритма и оценке времени его работы.
	
Подсчет нижних оценок сводится к нахождению общего вида всех алгоритмов данного класса, нахождению вещественнозначных параметров, однозначное определяющих работу алгоритма, построению выражения, зависящего от указанных параметров и определяющего математическое ожидание времени работы алгоритма, и, наконец, минимизации времени работы. При отсутствии аналитического решения, задача решается численным моделированием. 
\subsection{Black-box оптимизация задачи Needle}

Задача представлена оптимизирующему алгоритму, представленному в виде черного ящика, к  которому можно задавать запросы, и он возвращает либо 1, если запрос  совпал с загаданной строкой, либо 0 в обратном случае.

\section{Выводы по главе 2}



